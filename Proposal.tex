% Options for packages loaded elsewhere
\PassOptionsToPackage{unicode}{hyperref}
\PassOptionsToPackage{hyphens}{url}
%
\documentclass[
]{article}
\usepackage{amsmath,amssymb}
\usepackage{lmodern}
\usepackage{iftex}
\ifPDFTeX
  \usepackage[T1]{fontenc}
  \usepackage[utf8]{inputenc}
  \usepackage{textcomp} % provide euro and other symbols
\else % if luatex or xetex
  \usepackage{unicode-math}
  \defaultfontfeatures{Scale=MatchLowercase}
  \defaultfontfeatures[\rmfamily]{Ligatures=TeX,Scale=1}
\fi
% Use upquote if available, for straight quotes in verbatim environments
\IfFileExists{upquote.sty}{\usepackage{upquote}}{}
\IfFileExists{microtype.sty}{% use microtype if available
  \usepackage[]{microtype}
  \UseMicrotypeSet[protrusion]{basicmath} % disable protrusion for tt fonts
}{}
\makeatletter
\@ifundefined{KOMAClassName}{% if non-KOMA class
  \IfFileExists{parskip.sty}{%
    \usepackage{parskip}
  }{% else
    \setlength{\parindent}{0pt}
    \setlength{\parskip}{6pt plus 2pt minus 1pt}}
}{% if KOMA class
  \KOMAoptions{parskip=half}}
\makeatother
\usepackage{xcolor}
\IfFileExists{xurl.sty}{\usepackage{xurl}}{} % add URL line breaks if available
\IfFileExists{bookmark.sty}{\usepackage{bookmark}}{\usepackage{hyperref}}
\hypersetup{
  pdftitle={Project Proposal},
  pdfauthor={Haolin Zhong (hz2771), Shaocong Zhang (sz3030), Yuxuan Wang (yw3608), Boqian Li (bl2898)},
  hidelinks,
  pdfcreator={LaTeX via pandoc}}
\urlstyle{same} % disable monospaced font for URLs
\usepackage[margin=1in]{geometry}
\usepackage{longtable,booktabs,array}
\usepackage{calc} % for calculating minipage widths
% Correct order of tables after \paragraph or \subparagraph
\usepackage{etoolbox}
\makeatletter
\patchcmd\longtable{\par}{\if@noskipsec\mbox{}\fi\par}{}{}
\makeatother
% Allow footnotes in longtable head/foot
\IfFileExists{footnotehyper.sty}{\usepackage{footnotehyper}}{\usepackage{footnote}}
\makesavenoteenv{longtable}
\usepackage{graphicx}
\makeatletter
\def\maxwidth{\ifdim\Gin@nat@width>\linewidth\linewidth\else\Gin@nat@width\fi}
\def\maxheight{\ifdim\Gin@nat@height>\textheight\textheight\else\Gin@nat@height\fi}
\makeatother
% Scale images if necessary, so that they will not overflow the page
% margins by default, and it is still possible to overwrite the defaults
% using explicit options in \includegraphics[width, height, ...]{}
\setkeys{Gin}{width=\maxwidth,height=\maxheight,keepaspectratio}
% Set default figure placement to htbp
\makeatletter
\def\fps@figure{htbp}
\makeatother
\setlength{\emergencystretch}{3em} % prevent overfull lines
\providecommand{\tightlist}{%
  \setlength{\itemsep}{0pt}\setlength{\parskip}{0pt}}
\setcounter{secnumdepth}{-\maxdimen} % remove section numbering
\ifLuaTeX
  \usepackage{selnolig}  % disable illegal ligatures
\fi

\title{\textbf{Project Proposal}}
\author{Haolin Zhong (hz2771), Shaocong Zhang (sz3030), Yuxuan Wang
(yw3608), Boqian Li (bl2898)}
\date{}

\begin{document}
\maketitle

\center{

# Movie Recommender

a demo movie recommendation system

}

\hypertarget{motivation}{%
\subsection{Motivation}\label{motivation}}

In the era of information overload, pushing to users information
matching their preference has been a valuable application and a major
challenge in the field of data science. We proposed to explore the movie
lens dataset, a dataset consisting of users' rating and tagging towards
movies, and implement a demo movie recommendation system.

\hypertarget{the-intended-final-products}{%
\subsection{The intended final
products}\label{the-intended-final-products}}

\hypertarget{models}{%
\subsubsection{Models}\label{models}}

\begin{itemize}
\item
  Item-similarity based recommendation
\item
  User-similarity based recommendation
\item
  Collaborative filtering (Latent factor model (SVD)) based
  recommendation
\end{itemize}

\hypertarget{a-shiny-app}{%
\subsubsection{A shiny app}\label{a-shiny-app}}

\begin{itemize}
\tightlist
\item
  Recommends user movie after the user types his favorite 3 movies
\end{itemize}

\hypertarget{the-anticipated-data-sources}{%
\subsection{The anticipated data
sources}\label{the-anticipated-data-sources}}

\begin{itemize}
\item
  \href{https://grouplens.org/datasets/movielens/latest/}{MoiveLens
  dataset}
\item
  Movie information scaraped from \href{}{IMDB}
\end{itemize}

\hypertarget{the-planned-analyses-visualizations-coding-challenges}{%
\section{The planned analyses/ visualizations / coding
challenges}\label{the-planned-analyses-visualizations-coding-challenges}}

\hypertarget{planned-analyses}{%
\subsection{Planned analyses:}\label{planned-analyses}}

\begin{itemize}
\item
  Which movies have the highest ratings? Are there any significant
  differences in the scores of different categories of films? Is there
  any relationship between the film score and the year of film release?
  Is there any relationship between film rating and comment time?
\item
  Test the average ratings of the users, to investigate whether the
  rating criteria are different.(ANOVA)
\item
  Average ratings over years -- to investigate whether ratings are
  different because of the year the movie displayed.
\item
  Construct a linear model to predict the underlying rating of a
  specific user. Preference on specific genre. Whether there is a
  difference between genre and ratings.
\end{itemize}

\hypertarget{visualizations}{%
\subsection{Visualizations:}\label{visualizations}}

\begin{itemize}
\tightlist
\item
  Boxplot regarding the rating and years; boxplot regarding the genre
  and ratings; maybe a linear regression graph to show a specific user's
  rating on genre
\end{itemize}

\hypertarget{coding-challenges}{%
\subsection{Coding challenges:}\label{coding-challenges}}

\begin{itemize}
\item
  To acheive collaborative filtering (Latent factor model (SVD)) based
  recommendation, we may construct a M*N matrix to calculate the
  item-similarity and the user-similarity. We may also pick up the most
  frequent tags from the movie and used the tag to recommend the similar
  movies.
\item
  SVD
\end{itemize}

\hypertarget{the-planned-timeline}{%
\subsubsection{The planned timeline}\label{the-planned-timeline}}

\begin{longtable}[]{@{}ll@{}}
\toprule
Date & Task \\
\midrule
\endhead
Nov 4 & Brainstorm \\
Nov 9 & Finish draft proposal \\
Nov 13 & Submit formal proposal \\
Nov 15 & Assign tasks \\
Nov 16 - 19 & Project review meeting \\
Nov 30 & Finish coding part \\
Dec 7 & Construct website \\
Dec 11 & Report \& Webpage and screencast \& Peer asessment \\
Dec 14 & Presentation practice \\
Dec 16 & ``In class'' discussion \\
\bottomrule
\end{longtable}

\end{document}
